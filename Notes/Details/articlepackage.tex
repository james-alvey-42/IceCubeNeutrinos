\label{articlepackage}


\usepackage{soul}
%\usepackage{hyperref}
\usepackage{multicol}

\usepackage{setspace}
%\doublespacing

\usepackage[T1]{fontenc}
\usepackage{ifthen}


\usepackage{bm}
\usepackage{bibentry}
\usepackage{subcaption}
\usepackage{wrapfig}
\usepackage{amsmath}
\numberwithin{equation}{section}
\numberwithin{figure}{section}
\numberwithin{table}{section}
\numberwithin{footnote}{section}
\usepackage{mathtools}
\usepackage[inline]{enumitem}
\usepackage{booktabs}
%\usepackage[usenames,dvipsnames,pdftex]{xcolor}
\usepackage{tikz}
\usetikzlibrary{backgrounds,shapes,arrows,positioning,calc,snakes,fit}
\usepgflibrary{decorations.markings}
\usepackage{framed}
\usepackage{todonotes}
\newcommand{\mytodo}[1]{\todo[bordercolor=white, color=SAEblue!40!white]{\small #1}}

% \setcounter{section}{-1}

\usepackage{graphicx} % standard package
\usepackage{todonotes} % standard package
\usepackage{amsmath} % standard package
\DeclareMathOperator{\sech}{sech}
\newcommand*\diff{\mathop{}\!\mathrm{d}}
\newcommand{\txtd}{\textrm{d}}
\usepackage{amssymb} % useful for double backed letter functions
%%%%%%%%%%%%%%%%%%%%%%%%%%%%%%%%%%%%
\usepackage{amsthm} % used to define theorem objects with command \begin{theorem} etc.
\newtheorem{theorem}{Theorem}[section]
\newtheorem{example}{Example}[subsection]
\newtheorem*{definition}{Definition}
%%%%%%%%%%%%%%%%%%%%%%%%%%%%%%%%%%%%
\usepackage[numbers, sort&compress]{natbib}
\bibliographystyle{apsrev} % bibliography package and style
\renewcommand{\bibfont}{\small}
\renewcommand{\citenumfont}[1]{\textbf{#1}}
\renewcommand{\bibnumfmt}[1]{[\color{darkblue}\textbf{#1}\color{black}]}
%%%%%%%%%%%%%%%%%%%%%%%%%%%%%%%%%%%%
\usepackage{listings} % code listing and options
\usepackage{color}
%New colors defined below
\definecolor{codegreen}{rgb}{0,0.6,0}
\definecolor{codegray}{rgb}{0.5,0.5,0.5}
\definecolor{codepurple}{rgb}{0.58,0,0.82}
\definecolor{backcolour}{rgb}{0.95,0.95,0.92}
%Code listing style named "mystyle"
\lstdefinestyle{mystyle}{
  backgroundcolor=\color{backcolour},   commentstyle=\color{codegreen},
  keywordstyle=\color{magenta},
  numberstyle=\tiny\color{codegray},
  stringstyle=\color{codepurple},
  basicstyle=\footnotesize,
  breakatwhitespace=false,
  breaklines=true,
  captionpos=b,
  keepspaces=false,
  numbers=right,
  numbersep=4pt,
  showspaces=false,
  showstringspaces=false,
  showtabs=false,
  tabsize=2
}
\lstset{style=mystyle}

\usepackage{tensor}

\usepackage[flushmargin, hang]{footmisc}
  \addtolength{\footnotesep}{1mm}
  \setlength{\footnotemargin}{1em}
  \renewcommand{\thefootnote}{\textbf{\arabic{footnote}}}
  \renewcommand\footnoterule{{\hrule height 0.2pt}}

\captionsetup[figure]{labelsep=quad, labelfont=bf, textfont=it, width=0.8\linewidth}
\captionsetup[table]{labelsep=quad, labelfont=bf, textfont=it, width=0.8\linewidth}


\usepackage{sectsty}
\sectionfont{\color{darkblue}\centering \large \textsc}
\subsectionfont{\color{darkblue}\centering \normalsize \textit}
\subsubsectionfont{\color{darkblue}\centering \small \textit}
\renewcommand\thesection{\arabic{section}}
\renewcommand\thesubsection{\arabic{section}.\arabic{subsection}}
\renewcommand\thefigure{\arabic{section}.\arabic{figure}}
\renewcommand\theequation{{\color{SAEblue}\arabic{section}.\arabic{equation}}}
\usepackage{color}
\definecolor{SAEblue}{rgb}{0, .62, .91}
\definecolor{linkgreen}{RGB}{11, 102, 35}
\definecolor{darkblue}{rgb}{.11, .102, .35}

\usepackage[colorlinks]{hyperref}
\hypersetup{colorlinks=true, urlcolor=black, citecolor=linkgreen, runcolor=black, menucolor=black, filecolor=black, anchorcolor=black, linkcolor=black}
\theoremstyle{definition}


\theoremstyle{definition}

\renewcommand\vec{\mathbf}
\newcommand{\normord}[1]{\raisebox{0.5pt}{:}\,#1\,\raisebox{0.5pt}{:}}
\newcommand{\dagg}{^{\dagger}}
\newcommand{\pr}{^{\prime}}
\newcommand{\nhat}{\hat{\bm{n}}}
\newcommand{\hamilt}{\mathcal{H}}
\newcommand{\mA}{\mathcal{A}}
\newcommand{\mW}{\mathcal{W}}
\newcommand{\mN}{\mathcal{N}}
\newcommand{\mD}{\mathcal{D}}
\newcommand{\mS}{\mathcal{S}}
\newcommand{\mL}{\mathcal{L}}
\newcommand{\mC}{\mathcal{C}}
\newcommand{\mO}{\mathcal{O}}
\newcommand{\mM}{\mathcal{M}}
\newcommand{\mT}{\mathcal{T}}
\newcommand{\mZ}{\mathcal{Z}}
\newcommand{\mR}{\mathcal{R}}
\newcommand{\II}{\mathbb{I}}
\newcommand{\RR}{\mathbb{R}}
\newcommand{\ZZ}{\mathbb{Z}}
\newcommand{\CC}{\mathbb{C}}
\newcommand{\FF}{\mathbb{F}}
\newcommand{\lie}[1]{\mathcal{L}\left(#1\right)}
\newcommand{\set}[1]{\left\{#1\right\}}
\newcommand{\SO}[1]{\textrm{SO}\left(#1\right)}
\newcommand{\SU}[1]{\textrm{SU}\left(#1\right)}
\newcommand{\Orth}[1]{\textrm{O}\left(#1\right)}
\newcommand{\Uni}[1]{\textrm{U}\left(#1\right)}
\newcommand{\paraskip}{\vspace{10pt}}
\newcommand{\del}{\partial}
\newcommand{\TeG}{\mathcal{T}_e(\mathscr{G})}
\newcommand{\TpM}{\mathcal{T}_p(\mathcal{M})}
\newcommand{\TpMs}{\mathcal{T}^{\star}_p(\mathcal{M})}
\newcommand{\etamn}[1]{\eta#1{\mu \nu}}
\newcommand{\upd}[1]{\text{d}#1 \,}
\newcommand{\ud}{\text{d}}
\newcommand{\group}{\mathscr{G}}
\newcommand{\alge}{\mathfrak{g}}
\newcommand{\twobytwo}[4]{\begin{pmatrix}#1&#2 \\ #3&#4 \end{pmatrix}}
\newcommand{\thrbythr}[3]{\begin{pmatrix}#1 \\ #2 \\ #3\end{pmatrix}}
\newcount\colveccount
\newcommand*\colvec[1]{
        \global\colveccount#1
        \begin{pmatrix}
        \colvecnext
}
\def\colvecnext#1{
        #1
        \global\advance\colveccount-1
        \ifnum\colveccount>0
                \\
                \expandafter\colvecnext
        \else
                \end{pmatrix}
        \fi
}

\newenvironment{Figure}
  {\par\medskip\noindent\minipage{\linewidth}}
  {\endminipage\par\medskip}

\newcommand{\Abs}[1]{\left| #1 \right|}
\newcommand{\tr}{\text{Tr}}

\renewcommand\labelitemi{\raisebox{0.25ex}{\tiny$\bullet$}}
\setenumerate{label=\color{SAEblue}{\textbf{\arabic*}}\color{black})}


\usepackage[most]{tcolorbox}
\newtcolorbox{titlebox}{arc=0mm,auto outer arc, colback=blue!5!white,colframe=black!75!black,leftrule=0pt,rightrule=0pt,toprule=1pt,bottomrule=1pt}

\newtcolorbox{subbox}{arc=0mm,auto outer arc, colback=white!5!white,colframe=black!75!black,leftrule=0pt,rightrule=0pt,toprule=0pt,bottomrule=1pt}

\newtcolorbox{examplebox1}{breakable,enhanced,arc=0mm,auto outer arc, colback=black!5!white,colframe=black!75!black,leftrule=1pt,rightrule=0pt,toprule=0pt,bottomrule=0pt,left=0mm,right=0mm}

\newtcolorbox{examplebox2}{breakable,enhanced,arc=0mm,auto outer arc, colback=white!5!white,colframe=black!75!black,leftrule=1pt,rightrule=0pt,toprule=0pt,bottomrule=0pt,left=0mm,right=0mm}
